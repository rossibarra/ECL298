\documentclass[]{article}
\usepackage{hyperref}
\begin{document}

\title{ECL 298 Syllabus}
\maketitle

%lead on 1 2nd on 3
%writes a paragraph
% lead write panel summary

%rank 

%syllabus, assignment, lecture
%assignment: *NIX terminal, linux commands, sign in to cluster: HW screen shot

% 30% proposal & 10% review
% 30% workshop
% 30% discussion

%putty, cygwin, 

%20 individuals
%2x coverage each
%work on cluster

\section*{Instructors}

\href{mailto:micmiller@ucdavis.edu}{Mike Miller}\\
Meyer 2241\\

\noindent \href{mailto:rossibarra@ucdavis.edu}{Jeff Ross-Ibarra}\\
262 Robbins Hall\\

\noindent \href{mailto:awhitehead@ucdavis.edu}{Andrew Whitehead}\\
4121 Meyer Hall\\

\subsubsection*{TA}\\
\href{mailto:adurvasula@ucdavis.edu}{Arun Durvasula}\\
262 Robbins Hall\\

\noindent Office hours for all instructors are by appointment.

\section*{Workshops}
The workshop section of the class will investigate some basic population genomics questions in an endangered wild rice population from Costa Rica. Data analysis will be assigned as homework each week, and students are expected to come to class with a laptop to work on the data.  

\section*{Journal discussion}
Each week we will read two journal papers. A group of three students will be assigned to lead the discussion on these papers.

\section*{Proposal}
You will write an ecological genomics grant proposal, following the guidelines for the NSF DDIG.  The proposal abstract is due on 2/4, and the full proposal due 2/25. \\

\subsection*{Panel Review}
Each student will be assigned to be primary reviewer on 1 proposal and secondary reviewer on 3 proposals. The lead reviewer will write the panel summary for each proposal.

\section*{Grading}
Your grade will be 30\% proposal, 10\% proposal review, 30\% workshop, and 30\% discussion.

\section*{Schedule}
\begin{enumerate}

\item Lectures 1
\begin{itemize}
\item 1/5 Introduction to sequencing 
\item 1/7 Experimental design 
	\begin{itemize}
	\item HW: UNIX tutorial, Farm login
	\end{itemize}
\end{itemize}

\item Workshop/Reading
	\begin{itemize}
	\item 1/12 Workshop: sequence data
		\begin{itemize}
		%\item sequence data, data quality issues, read mapping issues
		\item HW: map reads, describe data
		\end{itemize}
	\item 1/14 Reading: comparative genomics
\end{itemize}

\item Workshop/Reading
	\begin{itemize}
	\item 1/19 MLK day: no class
	\item 1/21 Reading: genomic basis of adaptation, genotype-phenotype mapping
\end{itemize}

\item Workshop/Reading
	\begin{itemize}
	\item 1/26 Workshop: genetic diversity
		\begin{itemize}
%		\item SNP calling, genotype likelihoods, ANGSD intro
		\item HW: estimate diversity statistics		\end{itemize}
	\item 1/28 Reading: genomic basis of adaptation, genome scans
\end{itemize}

\item Workshop/Reading
	\begin{itemize}
	\item 2/2 Workshop: demography and selection
		\begin{itemize}
%		\item SFS, selective sweeps
		\item HW: SFS, genome scan		
		\end{itemize}
	\item 2/4 Reading: demography and admixture
		\begin{itemize}
%		\item SFS, selective sweeps
		\item proposal abstracts due		
		\end{itemize}
\end{itemize}

\item Workshop/Reading
	\begin{itemize}
	\item 2/9 Workshop: admixture and population structure
		\begin{itemize}
	%	\item STRUCTURE theory, $F_{ST}$, 2D SFS
		\item HW: admixture, $F_{ST}$ outliers 	\end{itemize}
	\item 2/11 Reading: functional genomics and physiological ecology
\end{itemize}

\item Workshop/Reading
	\begin{itemize}
	\item 2/16 President's Day: no class
%		\begin{itemize}
%		\item RNA-seq methods
%		\item HW: compare differential expressed genes with differentiation, introgression
%		\end{itemize}
	\item 2/18 Reading: community genomics
\end{itemize}

\item Workshop/Reading
	\begin{itemize}
	\item 2/23 Workshop: RNAseq
		\begin{itemize}
		\item HW: identify differentially expressed genes
		\end{itemize}
	\item 2/25 Reading: disease ecology and environmental forensics
		\begin{itemize}
		\item proposals due
		\end{itemize}
\end{itemize}

\item Panel discussions

\item Panel discussions

\end{enumerate}

\end{document}